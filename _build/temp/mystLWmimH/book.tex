% Created with jtex v.1.0.20
\documentclass[a4paper,11pt,oneside]{book}
\usepackage[top=2cm, bottom=2cm, left=2cm, right=2cm]{geometry}
\usepackage[T1]{fontenc}
\usepackage[utf8]{inputenc}
\usepackage{lmodern}
\usepackage{graphicx}
\usepackage{caption}
\usepackage{natbib}
\usepackage{xcolor}
\usepackage{changepage}
\usepackage{framed}
\usepackage{hyperref}
\usepackage{amssymb}
\bibliographystyle{abbrvnat}

% Make list items more compact
\usepackage{enumitem}
\setlist[itemize]{noitemsep, topsep=0pt}

%%%%%%%%%%%%%%%%%%%%%%%%%%%%%%%%%%%%%%%%%%%%%%%%%%
%%%%%%%%%%%%%%%%%%%%  imports  %%%%%%%%%%%%%%%%%%%
\usepackage{url}
%%%%%%%%%%%%%%%%%%%%%%%%%%%%%%%%%%%%%%%%%%%%%%%%%%


\hypersetup{
  colorlinks,
  linkcolor={black},
  citecolor={black},
  urlcolor={black}
}

% Style quotes
\definecolor{darkblue}{rgb}{0.0, 0.0, 0.55}
\definecolor{quoteshade}{rgb}{0.95, 0.95, 1}
\renewenvironment{quote}{%
  \def\FrameCommand{%
    \hspace{1pt}%
    {\color{darkblue}\vrule width 2pt}%
    {\color{quoteshade}\vrule width 4pt}%
    \colorbox{quoteshade}%
  }%
  \MakeFramed {\advance\hsize-\width \FrameRestore}%
  \noindent\hspace{-8pt}% disable indenting first paragraph
  \begin{adjustwidth}{0pt}{0pt}% adjust as needed
  \vspace{2pt}\vspace{2pt}%
}
{%
  \vspace{2pt}\end{adjustwidth}\endMakeFramed%
}

\title{\Huge \textbf{Chemical Data, Discovery and Design}}
\author{\textsc{Sam Chong and Joe Forth}}

\begin{document}
\sloppy

\frontmatter
\maketitle

\tableofcontents
% \listoffigures
% \listoftables

\mainmatter

% \include{preface.tex}
% \include{acknowledgements.tex}

\href{https://canvas.liverpool.ac.uk/courses/85632}{Canvas site}

This book hopefully acts as the main resource for content for the first half of CHEM502.

This book was originally written by Sam Chong and is now maintained and updated by Joe Forth. Please let me (Joe - \href{mailto:j.forth@liverpool.ac.uk}{j.forth@liverpool.ac.uk}) know if you spot any errors, cannot access any of the expected content, have suggestions for additional content or resources to include.

You can access the GitHub repository containing the markdown files and notebooks \href{https://github.com/joeforth/joebook}{here}. The code is available in the \texttt{book} directory and is broken down into topics.

There is a readme and environment yaml/requirements files in the root directory of the repository which should hopefully enable you create a conda environment in which you can run the notebooks.

\tableofcontents
\begin{framed}
\textbf{Warning}\\
The book is under active development. Content will be added throughout the semester.
\end{framed}



\end{document}
